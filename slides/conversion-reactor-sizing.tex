\begin{frame}\frametitle{\textbf{Chapter 2}: Major ideas for this section}
	\begin{itemize}
		\item[A] Definition of conversion
		\item[B] Reactor design \emph{for a specified conversion}
		\item[C] Graphical interpretation/solution of CSTR and PFR design equations
	\end{itemize}
\end{frame}

\begin{frame}\frametitle{Conversion}
	Consider $$aA + bB \longrightarrow cD + dD$$

	Consider species $A$ to be the \textbf{basis} (usually we pick the one which gets consumed to completion first, i.e. the limiting reagent). 
	
	\vspace{12pt}
	
	Useful to express {\color{purple}{extent of reaction}} in terms of {\color{purple}{\emph{conversion}}} of $A$: 
	
	$$X_A = \frac{\text{moles }A\text{ reacted}}{\text{moles } A\text{ fed}}$$ 
	
	\vspace{12pt}
	{\small To simplify notation, use $X$ (without subscript).}
\end{frame}

\begin{frame}\frametitle{Conversion}
	\textbf{Remarks}
	\begin{itemize}
		\item	In a batch reactor, $X$ is a \emph{function of time}
		\begin{itemize}
			\item	For irreversible reaction, $X\rightarrow1$ as $t\rightarrow\infty$
			\item	For reversible reaction, $X\rightarrow X_e$ (equilibrium conversion) as $t\rightarrow\infty$
		\end{itemize}
		\vspace{12pt}
		\item	In flow reactors (CSTR and PFR) the $X$ is a \emph{function of volume}, which reflects the amount of time reactants spend in the reactor
	\end{itemize}
\end{frame}

\begin{frame}\frametitle{Design equations in terms of conversion}
	In this section we consider rewriting the previously derived equations from Chapter 1 (called the design equations) in terms of conversion.
	
	We had previously used concentration $C_A$ and molar flow $F_A$. Now we will use {\color{purple}{$X$, conversion}}.
	
	\vspace{12pt}
	We consider our 4 reactors {\small (numbers refer to Fogler)}:
	\vspace{6pt}
	\begin{enumerate}
		\item[2.2]	Batch
		\item[2.3.1]	CSTR
		\item[2.3.2]	PFR
		\item[2.3.3]	PBR
	\end{enumerate}
\end{frame}

%\section{Batch}
\begin{frame}\frametitle{Batch systems}
	\vspace{12pt}
	First express $N_A$ in terms of $X$
	
	\begin{align*}
		\left(\begin{array}{c}
			\text{moles } A \\
			\text{reacted}
		\end{array}\right)
		&=
		\left(\begin{array}{c}
			\text{moles }A\\
			\text{fed}
		\end{array}\right)
		\cdot
		\left(\frac{\text{moles }A\text{ reacted}}{\text{moles }A\text{ fed}}\right)\\
		&= N_{A0} \cdot X\\
	\end{align*}
	\begin{align*}
		\left(\begin{array}{c}
			\text{moles }A\text{ at}\\
			\text{time }t
		\end{array}\right)
		&=
		\left(\begin{array}{c}
			\text{initial}\\
			\text{moles }A
		\end{array}\right)
		-
		\left(\begin{array}{c}
			\text{moles }A\\
			\text{consumed}
		\end{array}\right)
		\\
		N_A &= N_{A0} - N_{A0}\cdot X \\
		N_A &= N_{A0}(1 - X)
	\end{align*}
\end{frame}

\begin{frame}\frametitle{Batch systems}
	\begin{itemize}
		\item	Recall the mole balance equation (assumptions?):
		\begin{align*}
			\frac{dN_A}{dt} &= r_AV\\
			-N_{A0}\frac{dX}{dt} &= r_AV\\
			N_{A0}\frac{dX}{dt} &= -r_AV
		\end{align*}
		\item	Integral form: $$\int_0^tdt = t = N_{A0}\int_0^X\frac{dX}{-r_AV}$$
	\end{itemize}
	{\color{myOrange}{Design feature for batch systems: \emph{time}}}
\end{frame}

%\section{Flow Systems}
\begin{frame}\frametitle{Flow systems: CSTRs, PFRs and PBRs}
	\begin{align*}
		F_{A0}X &= \left(\frac{\text{moles }A\text{ fed}}{\text{time}}\right)\cdot\left(\frac{\text{moles }A\text{ reacted}}{\text{moles }A\text{ fed}}\right)\\
		F_{A0}X &= \frac{\text{moles }A\text{ reacted}}{\text{time}}
	\end{align*}
	\vspace{-36pt}
	\begin{columns}[t]
		\column{1.00\textwidth}
			\begin{align*}
				\quad\left(\begin{array}{c}
					\text{molar rate of}\\A\text{ leaving}
				\end{array}\right)
				&=
				\left(\begin{array}{c}
					\text{molar rate}\\\text{of }A\text{ fed}
				\end{array}\right)
				-
				\left(\begin{array}{c}
					\text{molar rate of}\\A\text{ consumed}
				\end{array}\right)
				\\
				F_A &= F_{A0} - F_{A0}X\\
				F_A &= F_{A0}(1 - X)
			\end{align*}
		\column{0.10\textwidth}
	\end{columns}
	\vspace{12pt}
	{\color{myOrange}{Draw a picture here $\rightarrow$}}
\end{frame}

%\subsection{CSTR}
\begin{frame}\frametitle{CSTR}
	\begin{itemize}
		\item	Recall the mole balance equation (what were the assumptions?):
				\begin{align*}
					V &= \frac{F_{A0} - F_A}{-r_A}\\
					V &= \frac{F_{A0} - F_{A0}(1 - X)}{-r_A} \\
					V &= \frac{F_{A0}X}{-r_A}
				\end{align*}
		\item	$r_A$: taken inside the reactor = exit conditions
	\end{itemize}
	{\color{myOrange}{Design feature for CSTR systems: \emph{volume}}}
\end{frame}

%\subsection{PFR}
\begin{frame}\frametitle{PFR}
	Recall the mole balance equation (what were the assumptions?):
	$$\frac{dF_A}{dV} = r_A$$

	But $F_A = F_{A0}(1 - X)$
	\begin{align*}
		-F_{A0}\frac{dX}{dV} &= r_a\\
		F_{A0}\frac{dX}{dV}&=-r_A
	\end{align*}

	Integral form:
	\vspace{-6pt}
	$$\int_0^VdV = V = F_{A0}\int_0^X\frac{dX}{-r_A}$$
	{\color{myOrange}{Design feature for PFR systems: \emph{volume}}}
\end{frame}

%\subsection{PBR}
\begin{frame}\frametitle{PBR}
	Follows a similar derivation:
	
	$$F_{A0} \frac{dX}{dW} = -r_A'$$ 
	
	Integral form: 
	
	$$W = F_{A0}\int_0^X\frac{dX}{-r_A'}$$
	
	\vspace{24pt}
	{\color{myOrange}{Design feature for PBR systems: \emph{catalyst weight}}}
\end{frame}

% PFR example: not done. Used the slides below instead
\begin{comment}
\begin{frame}\frametitle{PFR example}
	\begin{itemize}
		\item	{\color{blue}Ex 2-2} Soln:
		\item	(a) $$V = \frac{F_{A0}X}{(-r_A)_\text{exit}}$$ Given: $X = 0.8$

		From Table 2.3 $$\left(\frac{1}{-r_A}\right)_{X=0.8} = 20 \frac{\text{m}^3\cdot\text{s}}{\text{mol}}$$ Given: $F_{A0} = 0.4$mol/s $$\Rightarrow V = (0.4)(0.8)(20) = 6.4\text{m}^3$$
		\item	(b)
	\end{itemize}
\end{frame}

\begin{frame}[allowframebreaks]\frametitle{PFR example}

	Ex 2-3 Solution:

	(a)
	\begin{align*}
		&F_{A0}\frac{dX}{dV} = -r_A\\
		&V = \int_0^VdV = F_{A0}\int_0^X\frac{dX}{-r_A(X)}
	\end{align*}

	Trapezoidal formula: 

	$$\int_a^bf(x)dx\approx\frac{h}{2}(f_0 + 2f_1 + 2f_2 + \cdots + 2f_{n-1} + f_n)$$ 

	Simpson's rule for $n$ even: $$\int_a^bf(x)dx\approx\frac{h}{3}(f_0 + 4f_1 + 2f_2 + 4f_3 + \cdots + 2f_{n-2} + 4f_{n-1} + f_n)$$

	Applying Simpson's rule to Ex 2-3,
	\begin{align*}
		V &= F_{A0}\int_0^{0.8}\left(-\frac{1}{r_A(X)}\right)dX\\
		&= F_{A0}\frac{\Delta X}{3}\left[\left(-\left.\frac{1}{r_A}\right|_{X=0}\right) + 4\left(-\left.\frac{1}{r_A}\right|_{X=0.2}\right)\right.\\
		&\left.\hspace{5em}+ \cdots + \left(-\left.\frac{1}{r_A}\right|_{X=0.8}\right)\right]
	\end{align*}

	Using $\dfrac{1}{-r_A}$ values in Table 2-3 gives $V=2.165\,\text{m}^3$

	(b) $$V = \int_0^{0.8}\left(\frac{F_{A0}}{-r_A}\right)dX$$

	(c) Sketch of $X$ and $-r_A$ vs V?
	\begin{itemize}
		\item[(i)] Fix $X$, say $X = 0.2$
		\item[(ii)] Calculate $$V = F_{A0}\int_0^{0.2}\frac{dX}{-r_A}\approx218\text{dm}^3$$
		\item[(iii)] Increment $X$ and calculate new $V$, etc.
		\begin{table}
			\centering
			\begin{tabular}
				{|c|c|c|c} \hline $X$ & 0 & 0.2 & 0.4\\
				$-r_A$ & 0.45 & 0.30 &\\
				$V$ & 0 & 218 &\\
				\hline
			\end{tabular}
			$\leftarrow$ Given data,
		\end{table}
	\end{itemize}
\end{frame}

\begin{frame}\frametitle{PFR example}
	\begin{itemize}
		\item	{\color{blue}Ex 2-4} Comparison of CSTR and PFR sizes

		In general -

		For isothermal reactions of order $>$ zero, $-\frac{1}{r_A}$ vs $X$ curve is concave ??.

		\item	$\Rightarrow$ PFR will require a smaller volume than CSTR for same feed rate.
	\end{itemize}
\end{frame}
\end{comment}

\begin{frame}\frametitle{{\color{myRed}{\sout{Summary so far}}}\,\,{\large Apply to a 1st order system}}
	\textbf{Remark}: we generally obtain $-r_A$ vs $X$ from rate equation, e.g. $-r_A = kC_A$. Consider a flow reactor:
	\begin{align*}
		&F_A = F_{A0}(1 - X)\\
		&C_A = \frac{F_A}{q} = \frac{F_{A0}(1 - X)}{q}
	\end{align*}
	If $q = q_0$ {\color{myGreen}(under what conditions would this hold?)}:
	\begin{align*}
		C_A &= \frac{F_{A0}}{q_0}(1 - X)\\
		C_A &= C_{A0}(1 - X)
	\end{align*}
	and $$-r_A = k C_A = kC_{A0}(1 - X)$$
\end{frame}

\begin{frame}\frametitle{Example: data collected}
	\begin{center}
		\includegraphics[width=\textwidth]{\imagedir/reactors/example-2-2/raw-data.png}
	\end{center}
	\vfill
	\begin{itemize}
		\item	$F_{A0} = 0.4\,\text{mol.s}^{-1}$
		\item	Isothermal and constant pressure; gas-phase
		\item	We have \textbf{no idea what the reaction order is}
	\end{itemize}
	\vspace{96pt}
	
\end{frame}

\begin{frame}\frametitle{Example: plotted}
	\begin{center}
		\includegraphics[width=.8\textwidth]{\imagedir/reactors/example-2-2/empty-plot-modified.png}
	\end{center}
\end{frame}

\begin{frame}\frametitle{Example: plotted (with best-fit line added)}
	\begin{center}
		\includegraphics[width=.8\textwidth]{\imagedir/reactors/example-2-2/empty-plot.png}
	\end{center}
\end{frame}

\begin{frame}\frametitle{CSTR solution}
	\begin{center}
		\includegraphics[width=\textwidth]{\imagedir/reactors/example-2-2/CSTR-solution.png}
	\end{center}
\end{frame}

\begin{frame}\frametitle{PFR solution}
	\begin{center}
		\includegraphics[width=\textwidth]{\imagedir/reactors/example-2-2/PFR-solution.png}
	\end{center}
\end{frame}

\begin{frame}\frametitle{Simpson's rules}
	\begin{itemize}
		\item	Got 3 equally-spaced points; with spacing = $h$?
				$$ \int_{x_0}^{x_2} f(x) dx \approx \dfrac{h}{3}\Biggl[f(x_0) + 4f(x_1) + f(x_2) \Biggr]$$
		\vspace{12pt}
		\item	Got 4 equally-spaced points; with spacing = $h$?
				$$ \int_{x_0}^{x_3} f(x) dx \approx \dfrac{3h}{8}\Biggl[f(x_0) + 3f(x_1) + 3f(x_2) + f(x_3)\Biggr]$$
		\vspace{12pt}
		\item	\small See Appendix A for other formulas (more general)
	\end{itemize}
	
\end{frame}

\begin{frame}\frametitle{CSTR \emph{vs} PFR}
	\begin{center}
		\includegraphics[width=0.95\textwidth]{\imagedir/reactors/example-2-2/PFR-CSTR-difference.png}
	\end{center}
\end{frame}

\begin{frame}\frametitle{Profiles along the reactor: $r_A$}
	\begin{columns}[c]
		\column{0.80\textwidth}
		\begin{center}
			\includegraphics[width=1.05\textwidth]{\imagedir/reactors/example-2-2/rA-profile.png}
		\end{center}
		\column{0.20\textwidth}
			{\color{myOrange}{How was this found?}}
	\end{columns}
\end{frame}


\begin{frame}\frametitle{Profiles along the reactor: $X$}
	\begin{columns}[c]
		\column{0.80\textwidth}
			\begin{center}
				\includegraphics[width=0.95\textwidth]{\imagedir/reactors/example-2-2/X-profile.png}
			\end{center}
		\column{0.20\textwidth}
			{\color{myOrange}{How was this found?}}
	\end{columns}	
\end{frame}

%\section{Reactors in series}
% \begin{frame}\frametitle{CSTRs: lab-scale}
% 	\begin{columns}[t]
% 		\column{0.50\textwidth}
% 			\begin{center}
% 				\includegraphics[width=\textwidth]{\imagedir/reactors/CSTRs/CSTR-lab-GMD.jpg}
% 			\end{center}
% 		\column{0.50\textwidth}
% 			\begin{center}
% 				\includegraphics[width=\textwidth]{\imagedir/reactors/CSTRs/CSTR-lab-back-view-GMD.png}
% 			\end{center}
% 	\end{columns}
% \end{frame}

\begin{frame}\frametitle{Reactors in series: multiple CSTRs}
	South Africa, the Ergo tailings plant
	\begin{center}
		\includegraphics[width=\textwidth]{\imagedir/reactors/CSTRs/CSTRs-in-series-South-Africa.png}
	\end{center}
\end{frame}

\begin{frame}\frametitle{Flotation cells: Bolivia}
	\iftoggle{internal}{
		\begin{center}
			\includegraphics[width=\textwidth]{\imagedir/reactors/CSTRs/flotation-flickr-6343899995_e9ecd80533_b-not-licensed.jpg}
		\end{center}
	}{
		See \href{http://www.flickr.com/photos/minerasancristobal/6343899995/}{flotation cells in series} on flickr.com
	}
	
\end{frame}

\begin{frame}\frametitle{Reactors in series: multiple CSTRs}
	\begin{columns}[t]
		\column{0.99\textwidth}
			\begin{center}
				\includegraphics[width=1.1\textwidth]{\imagedir/reactors/CSTRs/CSTR-in-series.png}
			\end{center}
		\column{0.01\textwidth}
		% 			
	\end{columns}
\end{frame}

\begin{frame}\frametitle{CSTR solution}
	\begin{center}
		\includegraphics[width=\textwidth]{\imagedir/reactors/example-2-2/CSTR-solution.png}
	\end{center}
\end{frame}

\begin{frame}\frametitle{Rule for conversions in series}
	
	\begin{exampleblock}{}
		$$X_n = \frac{\text{\sout{total moles of A reacted, leaving reactor }}n}{\text{moles A fed to \textbf{first} reactor}}$$
	\end{exampleblock}
	\vspace{-12pt}
	
	\begin{exampleblock}{}
		$$X_n = \frac{\text{\small total moles of A reacted from start, up to reactor }n}{\text{moles A fed to \textbf{first} reactor}}$$
	\end{exampleblock}
	
	\vspace{-4pt}
	e.g. for 2 reactors in series
	\begin{align*}
		V_2 &= \frac{F_{A1} - F_{A2}}{-r_{A2}}\\
			&= \frac{F_{A0}(1 - X_1) - F_{A0}(1 - X_2)}{-r_{A2}}\\
			&= \frac{F_{A0}(X_2 - X_1)}{-r_{A2}}
	\end{align*}
\end{frame}

\begin{frame}\frametitle{}
	\begin{center}
		\includegraphics[width=\textwidth]{\imagedir/reactors/CSTRs/plot-two-in-series.png}
	\end{center}
\end{frame}


% \begin{frame}\frametitle{Reactors in series}
% 
% 	We will define conversion on basis of A fed to {\color{myGreen}first} reactor.
% 	\vspace{24pt}
% 	
% 	For example, for 2 reactors in series:
% 	\begin{columns}[t]
% 		\column{0.30\textwidth}
% 			
% 		\column{0.60\textwidth}
% 			Reactor 1 conversion: defined in the usual way
% 			\vspace{12pt}
% 			
% 			$$X_2 = \frac{\text{total moles of A reacted leaving reactor 2}}{\text{moles A fed to \textbf{first} reactor}}$$
% 	\end{columns}
% 		
% 	%Appropriate design equations for above configuration:
% \end{frame}
% 
% \begin{frame}\frametitle{Reactors in series: PFR then a CSTR}
% 	\begin{itemize}
% 		\item	{\color{myGreen}Reactor 1} $$V_1 = F_{A0}\int_0^{X_1}\frac{dX}{-r_A}$$
% 
% 		
% 		\item	{\color{myGreen}Reactor 2}
% 				\begin{align*}
% 					V_2 &= \frac{F_{A1} - F_{A2}}{-r_{A2}}\\
% 					&= \frac{F_{A0}(1 - X_1) - F_{A0}(1 - X_2)}{-r_{A2}}\\
% 					&= \frac{F_{A0}(X_2 - X_1)}{-r_{A2}}
% 				\end{align*}
% 
% 		%{\color{myGreen}Reactor 3} $$V_1 = F_{A0}\int_{X_2}^{X_3}\frac{dX}{-r_A}$$
% 	\end{itemize}
% \end{frame}

% \begin{frame}\frametitle{Reactors in series}
% 	\begin{itemize}
% 		\item	Follows from $$\frac{dF_A}{dV} = r_A,\quad F_A = F_{A0}(1 - X)$$ Graphically,
% 	\end{itemize}
% \end{frame}

% \begin{frame}\frametitle{Reactors in series}
% 	In general:
% 
% 	$$V_\text{CSTR} = \frac{F_{A0}(X_\text{Aout} - X_\text{Ain})}{(-r_A)_\text{out}}$$ $$V_\text{PFR} = F_{A0}\int_{X_\text{in}}^{X_\text{out}}\frac{dX}{-r_A}$$
% 	
% \end{frame}

% \begin{frame}\frametitle{}
% 	\todo{Add figure 2.6 or Levenspiel equivalent}
% \end{frame}

\begin{frame}\frametitle{Reactors in series: multiple CSTRs}
	\begin{itemize}
		\item	Consider $N$ CSTRs in series

		We observe that system approximates performance of a PFR of volume $$V_\text{PFR} \approx V_1 + V_2 + \cdots + V_N$$ Approximation improves as $N$ increases.
	\end{itemize}
\end{frame}

\begin{frame}\frametitle{Homework exercise}
	\begin{itemize}
		\item	Example on page 56 (F2011)
		\item	Example on page 62 (F2006)
	\end{itemize}
	%\todo{SHOW FIGURE HERE}
\end{frame}

\begin{frame}\frametitle{3 reactors in series}
	\begin{center}
		\includegraphics[width=\textwidth]{\imagedir/reactors/CSTRs/three-in-series-picture.png}
	\end{center}
\end{frame}

\begin{frame}\frametitle{3 reactors in series}
	\begin{center}
		\includegraphics[width=\textwidth]{\imagedir/reactors/CSTRs/three-in-series-plot.png}
	\end{center}
\end{frame}

%\section{Some further definitions}
\begin{frame}\frametitle{Some further definitions}
	\begin{itemize}
		\item	Space time, or residence time:
		$$\tau = \frac{V}{q_0}$$ time necessary to process one reactor volume of fluid based on entrance conditions
	\end{itemize}
\end{frame}
