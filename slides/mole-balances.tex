% 09 January 2013
\begin{frame}\frametitle{Slide credits and notes}
	\begin{itemize}
		\item	These slides were created by Dr. Mhaskar.
		\item	Minor modifications/additions by Kevin Dunn.
	\end{itemize}

	\vspace{6pt}
	\begin{itemize}
		\item	Words in {\color{purple}{purple}} are definitions
		\item	Words in \href{http://learnche.mcmaster.ca/3K4}{this colour} are hyperlinks
		\item	Other colours are to emphasize points
		\vspace{6pt}
		\hrule
		\vspace{6pt}
		\item	Version numbering: first page shows an integer version number, and refers to each revision; also includes the month and year for reference.
		\item	File names: \texttt{2013-3K4-class-01B.pdf} implies today, which is week 1 of the term, second class (\texttt{1A}, \texttt{1B}, \texttt{1C}, \texttt{2A}, \texttt{2B} \emph{etc})
		\item	Also used on course website
	\end{itemize}
\end{frame}

\begin{frame}\frametitle{What we are going to learn here}
	Review of chemistry and chemical engineering topics
	\begin{itemize}
		\item	Rate of reaction
		\item	Mole balances
		\item	Batch reactors
		\item	Continuous Tank Reactors (CSTRs)
		\item	Tubular reactors
		\item	Packed-bed reactors
	\end{itemize}
\end{frame}

\begin{frame}\frametitle{Rate of reaction}
	\begin{itemize}
		\item	 Chemical reactions are said to occur when species lose their identity, e.g. by decomposition, combination or isomerization.
		\begin{itemize}
			\item	$ A  \longrightarrow B + C $
			\item	$ A + B \longrightarrow C + D $
			\item	\includegraphics[width=.45\textwidth]{\imagedir/reactors/examples/Wikipedia-Rasveratrol_isomerization.png}\qquad\see{\href{http://en.wikipedia.org/wiki/Isomerization}{http://en.wikipedia.org/wiki/Isomerization}}
		\end{itemize}
		\item	 {\color{purple}{Rate of reaction}} is expressed as the rate of formation of products.
	\end{itemize}
\end{frame}

\begin{frame}\frametitle{Rate of reaction}
	By definition:
	\begin{exampleblock}{}
		\begin{center}
			{\Large $r_j$ is defined as the rate of formation of species $j$ per unit volume}
		\end{center}
	\end{exampleblock}
	\begin{itemize}
		\item	\only<1>{{\color{purple}{formation}} $\equiv$ {\color{purple}{generation}}}
				\only<2>{{\color{purple}{disappeared}} $\equiv$ {\color{purple}{consumed}}}
		\item	 Consider the reaction $$ A + B \longrightarrow C + D $$
		\begin{itemize}
			\item	\only<1>{$r_A$ is the number of moles of species $A$ {\bf formed} per unit time per unit volume $\left[\dfrac{\text{mol}}{\text{s.L}}\right]$ or $\left[\dfrac{\text{mol}}{\text{s.m}^3}\right]$}
					\only<2>{$-r_A$ (now a positive quantity) is the number of moles of species $A$ {\bf disappearing} per unit time per unit volume $\left[\dfrac{\text{mol}}{\text{s.L}}\right]$ or $\left[\dfrac{\text{mol}}{\text{s.m}^3}\right]$}
			\item	\only<1>{$r_A < 0$ in this case; yes, the sign is important for reaction rates}
					\only<2>{$-r_A > 0$ in this case}
		\end{itemize}
	\end{itemize}
\end{frame}

\begin{frame}[label=intensive-property]\frametitle{What affects $-r_A$ ?}
	\begin{itemize}
		\item	 {Reaction rate is an \emph{\color{purple}intensive property} $-$ function of concentration and temperature, but not size or configuration of reactor.}
		\item	Thermo recall: an {\color{purple}{intensive property}} does \textbf{not} depend on the size or scale of the system.

				These are great properties to deal with: they apply whether we are dealing with

			\begin{columns}[c]
				\column{0.50\textwidth}
					\begin{center}
						\includegraphics[width=1.0\textwidth]{\imagedir/reactors/examples/reactor-design-GMD-4-DETACLAD-Explosion-Clad.png}

						25m $\times$ 5m
					\end{center}
					%\see{Hydrometallurgy of Nickel and Cobalt 2009, The Metallurgical Society of the Canadian Institute of Mining, Metallurgy and Petroleum, Symposium}
					% Published in Hydrometallurgy of Nickel and Cobalt 2009, Proceedings of the International Symposium 39th Annual Hydrometallurgy Meeting
					% The Metallurgical Society of the Canadian Institute of Mining, Metallurgy and Petroleum, Pages 181-193,
					% DETACLAD Explosion Clad for Autoclaves and Vessels, John G.Banker
				\column{0.05\textwidth}
					or
				\column{0.45\textwidth}
					\begin{center}
						\includegraphics[width=.8\textwidth]{\imagedir/reactors/examples/Wikipedia-Cell-Average_prokaryote_cell-_en.svg.png}

						$\sim 1\text{ to }5\mu m$
					\end{center}
			\end{columns}
	\end{itemize}
\end{frame}

\begin{frame}\frametitle{What affects $-r_A$ ?}
	\begin{itemize}
		\item	Generally:
		\begin{enumerate}
			\item	temperature, $T$
			\item	concentration, $C_j$
			\item	pressure (for gas systems we generally use pressure instead of concentration), $P_j$
			\item	catalysts (we won't consider this for now)
		\end{enumerate}
		\item	Rate law is an algebraic expression of the form
				\[
					-r_A = \Biggl(k_A(T)\Biggr)\Biggl(f(C_A, C_B, \ldots)\Biggr)
				\]
	\end{itemize}
\end{frame}

\begin{frame}\frametitle{Analyzing $-r_A$ dependence}
	\begin{itemize}
		\item	 A bank lends money at the rate of 5\%. Does the interest depend on the amount borrowed?
		\item	 No, the interest rate is independent of the amount borrowed
	\end{itemize}
\end{frame}

\begin{frame}\frametitle{Analyzing $-r_A$ dependence}
	\begin{itemize}
		\item	Does the ``amount'' of reaction depend on the volume, i.e. the size of the system, \emph{etc}?
		\item	No, the reaction rate is independent of the volume.
		\item	The \textbf{number of moles} per unit volume determines concentration, which in turn determines the rate.
		\item	So there is a dependence on the concentration, but not on the volume.
	\end{itemize}
\end{frame}

% \begin{frame}
% 	\begin{itemize}
% 		\item	{In heterogeneous systems, rate of reaction is usually expressed in terms other than volume. e.g. for a gas-solid catalytic reaction $-r_A'$ denotes the number of moles of $A$ reacting per unit time per unit mass of catalyst. (mol/s$\cdot$g catalyst)}
% 	\end{itemize}
% \end{frame}

\begin{frame}{Examples of reaction rate expressions}
	For $$A\longrightarrow\text{products}$$ we might have
	$$-r_A = kC_A$$ or
	$$-r_A = kC_A^2$$ or
	$$-r_A = k\frac{k_1C_A}{1 + k_2C_A}$$
	\emph{and many others are possible}.

	\vspace{12pt}
	The last equation is common in bioreactor systems.
\end{frame}

\begin{frame}{How do we obtain these rate expressions?}
	%\textbf{Remark:} $r_A$ independent of the stoichiometry, but mole balances use the stoichiometry (see later)

	\begin{itemize}
		\item	Structure of the rate expression is guided by theory; trial and error testing
		\vspace{12pt}
		\item	We obtain the parameters from experiments (covered later)
			\begin{itemize}
				\item	e.g. $-r_A = kC_A^n$
				\item	determine $k$ and $n$ from experimental data
			\end{itemize}
	\end{itemize}
\end{frame}

\section{General mole balance equation}

\begin{frame}\frametitle{The general mole balance equation}
	We will use this over and over in the course. {\color{myOrange}{Let's understand this}}.

	\begin{exampleblock}{}
		A mole balance on species $j$ at any instant of time $t$, in an arbitrary system volume ($V_i$):
		\begin{center}
			\includegraphics[width=\textwidth]{\imagedir/reactors/mhaskar/ch1/MoleBalance5.pdf}
		\end{center}
	\end{exampleblock}
\end{frame}

\begin{frame}\frametitle{The general mole balance equation}

		A \textbf{mole balance} (not a mass balance) on species $j$ at any instant of time $t$:

		$$\text{In} + \text{Generated} - \text{Out} = \text{Accumulated}$$
		\scalebox{0.70}{
		\begin{minipage}
			{1.0\linewidth}
		\[
			\begin{bmatrix}
				\text{Rate of flow}\\
				\text{of }j\text{ into}\\
				\text{the system}\\
				\text{(moles/time)}
			\end{bmatrix}
			+
			\begin{bmatrix}
				\text{Rate of generation}\\
				\text{of }j\text{ by chemical}\\
				\text{reaction within}\\
				\text{the system}\\
				\text{(moles/time)}
			\end{bmatrix}
			-
			\begin{bmatrix}
				\text{Rate of flow}\\
				\text{of }j\text{ out of}\\
				\text{the system}\\
				\text{(moles/time)}
			\end{bmatrix}
			=
			\begin{bmatrix}
				\text{Rate of}\\
				\text{accumulation}\\
				\text{of }j\text{ within}\\
				\text{the system}\\
				\text{(moles/time)}
			\end{bmatrix}
		\]
		\end{minipage}
		} $$ F_{j0} + G_j - F_j = \frac{dN_j}{dt} $$
		\vspace{-4pt}
		\begin{itemize}
			\item	$G_j = \dfrac{\text{moles}}{\text{time}} = r_j \cdot V = \dfrac{\text{moles}}{(\text{time})(\text{volume})}\cdot\text{volume}$
			\item	{\color{myOrange}{{\small Why don't we have a term for ``consumed''?}}}
		\end{itemize}
\end{frame}

\begin{frame}\frametitle{The general mole balance equation}
	\begin{itemize}
		\item	We can subdivide the region into many subvolumes ($M$ of them), and for each subvolume $i = 1, 2, \ldots M$:
		\[
			\Delta G_{ji} = r_{ji}\Delta V_i
		\]
		$\Delta G_{ji}$ is the rate of generation of $j$ within subvolume $i$ (chosen to be small enough so that all variables are constant, and therefore the reaction rate is the same in the subvolume)
	\end{itemize}
\end{frame}

\begin{frame}\frametitle{The general mole balance equation}
	\begin{itemize}
		\item	Then $$G_j = \sum_1^M\Delta G_{ji} = \sum_1^M r_{ji}\Delta V_i$$
		 		an in the limit as $\Delta V_i\rightarrow0$ and $M\rightarrow\infty$
				$$G_j = \int_V r_j\,dV$$
				giving
				$$F_{j0} - F_j + \int_V r_j\,dV = \frac{dN_j}{dt}\eqno{\text{(1-4)}}$$
	\end{itemize}
\end{frame}

\begin{frame}\frametitle{The general mole balance equation}
	\begin{itemize}
		\item	{\color{red}{\textbf{Note}}}: $\displaystyle\int_V r_j\,dV \longleftarrow$ the $r_j$ term is a function of $V$ and cannot be taken out of the integral.
		\vspace{12pt}
		\item	Wait a minute: you said ``Reaction rate is an {\color{purple}{intensive property}}'' (slide \autoref{intensive-property}), not a function of the volume/size/scale of the system
		\begin{center}
			\includegraphics[width=0.6\textwidth]{\imagedir/reactors/examples/autoclave-GMD.png}
		\end{center}
	\end{itemize}
\end{frame}

\begin{frame}\frametitle{The general mole balance equation}
	\begin{itemize}
		\item	If all the system variables (temperature, pressure, concentration, \emph{etc}.) are uniform throughout the volume, $V$, \textbf{only then} can the reaction rate be taken out:
		\item	More correct to write:\qquad $r_j(V)$
		\begin{align*}
			G_j &=\int_V r_j(V)dV= r_j\int_V dV= r_j\cdot V\\
			\\
			\frac{\text{moles}}{\text{time}} &= \frac{\text{moles}}{(\text{time})(\text{volume})}\cdot{\text{volume}}
		\end{align*}
	\end{itemize}
\end{frame}

\begin{frame}\frametitle{The general mole balance equation}
	\begin{itemize}
		\item	Now, let's apply the balance equation to various reactor types 
	\end{itemize}
\end{frame}

% 10 January 2013
\section{Batch}
\begin{frame}\frametitle{Batch reactors}
	\begin{itemize}
		\item	No inflow or outflow.
		\item	Charge reactors with reactants, then close up.
		\item	What then?
	\end{itemize}
\end{frame}

\begin{frame}\frametitle{Batch systems}
	\begin{columns}[t]
		\column{0.50\textwidth}
			\begin{center}
				\includegraphics[width=\textwidth]{\imagedir/batch/batch-system.png}
			\end{center}
			\see{From Cecilia Rodrigues' M.A.Sc thesis, 2006, McMaster University, used with permission}
			
		\column{0.50\textwidth}
			\begin{center}
				\includegraphics[width=\textwidth]{\imagedir/batch/flickr-batch-reactor-2516220152_074fbbb489_o.jpg}
			\end{center}
			\see{Flickr: \href{http://www.flickr.com/photos/polapix/2516220152/}{\#2516220152}}
	\end{columns}
\end{frame}

\begin{frame}\frametitle{Batch systems}
	\iftoggle{internal}
	{
		\begin{center}
			\includegraphics[width=.98\textwidth]{\imagedir/reactors/CSTRs/fermenter-not-licensed.jpg}
		\end{center}
		\vspace{-16pt}
		\see{http://woodmoorbeer.org/Pages/sierra.html}
	}{See illustration at \href{http://woodmoorbeer.org/Pages/sierra.html}{http://woodmoorbeer.org/Pages/sierra.html}}
\end{frame}

\begin{frame}\frametitle{Batch systems}
	Recorded values from a single batch
	\begin{center}
		\includegraphics[width=0.85\textwidth]{\imagedir/examples/fmc/fmc-phases-4-trajectories.png}
	\end{center}
	\vspace{-18pt}
	\see{Speciality chemical manufacturer}
\end{frame}

\begin{frame}\frametitle{Batch reactors}
	Where are they used?
	\begin{itemize}
		\item	small scale products (low volumes)
		\item	usually extremely high value product: medicines, speciality chemicals
		\item	hard-to-make products
		\item	multiple steps in the ``recipe''
	\end{itemize}
\end{frame}

\begin{frame}\frametitle{Batch reactors}
	\begin{itemize}
		\item	Typically perfectly mixed, so $\displaystyle \int_V r_j(V)dV$ can be replaced by $r_j \cdot V$
		\item	The mole balance: start with the general equation, then simplify:
		\[
			\begin{aligned}
				F_{j0} - F_j + \int_V r_j(V) \,dV &= \frac{dN_j}{dt}\\
				\\
				\cancelto{0}{F_{j0}} - \cancelto{0}{F_j} + r_j \cdot V        &= \frac{dN_j}{dt}\\
			\end{aligned}
		\]
	\end{itemize}
\end{frame}

\begin{frame}\frametitle{Batch reactors}
	$$\frac{dN_j}{dt} = r_j \cdot V$$ 
	
	\textbf{Note:} $V$ is \emph{not} assumed to be constant here. It could be a function of time or of the extent of reaction (and indirectly a function of time).
	\vspace{12pt}
	\begin{exampleblock}{So for batch reactors:}
		$$\frac{dN_j(t)}{dt} = r_j(t) \cdot V(t)$$
	\end{exampleblock}
\end{frame}

\begin{frame}\frametitle{Batch example problem}
	$$A\longrightarrow2B$$
	with $-r_A = kC_A$ and $k = 0.23\text{min}^{-1}$ in a constant volume batch reactor.

	\vspace{12pt}
	We are also given the inlet concentration and volume: 
	$C_{A0} = 2$mol/L, $V=10$L. 
	
	\vspace{12pt}
	How long does it take to reduce the concentration of $A$ in reactor to 10\% of its initial value (i.e. a 90\% conversion)? 
	%Solution: {\bf First solve analytically} $$\frac{dN_A}{dt} = r_AV = -kC_AV$$ 
\end{frame}

\begin{frame}\frametitle{Batch example problem}
	
\end{frame}

\begin{frame}\frametitle{Batch example problem}
	
\end{frame}

\begin{comment}
\begin{frame}\frametitle{Solution 1.1}
	For const V,
	\begin{align*}
		&\frac{d(N_A/V)}{dt} = -kC_A\\
		&\frac{dC_A}{dt} = -kC_A\\
		&\Rightarrow \int_{C_{A0}}^{C_{Af}}\frac{dC_A}{C_A} = -k\int_0^{t_f}dt\\
		&\Rightarrow \ln\frac{C_{Af}}{C_{A0}} = -kt_f
	\end{align*}
\end{frame}

\begin{frame}
	\begin{align*}
		\Rightarrow t_f &= -\frac{1}{k}\ln\left(\frac{C_{Af}}{C_{A0}}\right)
	\end{align*}
	Now plug in the numbers
	\begin{align*}
		t_f &= -\frac{1}{0.23}\ln\left(\frac{0.2}{2}\right)\\
		&= 10\text{min}
	\end{align*}
\end{frame}
\end{comment}

\begin{comment}
\begin{frame}\frametitle{Problem 1.2}
	In a well mixed reactor, starting with a reactant A at concentration $C_{A0}$ and volume $V$, a reaction with the rate expression $r_a=-kC_A$ takes place. However, every $\Delta t$, a volume $V_0$ is removed first and fresh volume of $V_0$ (at concentration $C_{A0}$) is inserted into the reactor. Determine the steady state profile for the system (meaning the profile of $C_A(t)$ repeats every $\Delta t$)?
\end{frame}

\begin{frame}\frametitle{Solution 1.2}
	Consider the starting of a fresh interval at which the concentration of A in the reactor is $C_A^*$. Then, by a mole balance for a batch reactor (see before), one gets $$C_A(\Delta t)=C_A^*e^{{-k\Delta t}}$$ The number of moles removed=$$C_A^*e^{{-k\Delta t}}V_0$$
\end{frame}

\begin{frame}
	The number of moles added=$$C_{A0}V_0$$ If a steady state is to be achieved, $$C_A^*e^{{-k\Delta t}} (V-V_0)+C_{A0}V_0=C_A^*V$$ This yields $$C_A^*(V-e^{{-k\Delta t}}(V-V_0))=C_{A0}V_0$$
\end{frame}

\begin{frame}
	Further, $$C_A^*(V-e^{{-k\Delta t}}(V)+ V_0e^{{-k\Delta t}})=C_{A0}V_0$$ $$\Leftrightarrow C_A^*(V(1-e^{{-k\Delta t}})+V_0e^{{-k\Delta t}})=C_{A0}V_0$$ $$\Leftrightarrow C_A^*=\frac{C_{A0}V_0}{V(1-e^{{-k\Delta t}})+V_0e^{{-k\Delta t}}}$$

	Let us analyze for $\Delta t\rightarrow 0$. What does this mean?

	To do this, define $V_0/\Delta t=v_0$ (volumetric flow rate).
\end{frame}

\begin{frame}
	This gives $$C_A^*=\frac{C_{A0}v_0\Delta t}{V(1-e^{{-k\Delta t}})+v_0\Delta te^{{-k\Delta t}}}$$ Now taking the limit $\Delta t \rightarrow 0$, results in (L'Hospital's rule) $$C_A^*=\frac{C_{A0}v_0}{V(k)+v_0(1)}=\frac{C_{A0}}{1+k\tau}$$ Is this result expected? (In assignment 1, repeat this analysis, but `add' the fresh volume first and then remove the volume from the reactor)
\end{frame}
\end{comment}

\section{Continuous-flow}
\subsection{CSTR}

\begin{comment}
\begin{frame}\frametitle{\normalsize 1.4 Continuous-Flow Reactors\\1.4.1 Continuous Tank Reactors (CTR) }
	\begin{columns}
		\begin{column}{0.6\textwidth} 
			\vspace{-15mm}
			\includegraphics[width=1.7in, height=2.7 in]{\imagedir/reactors/mhaskar/ch1/CSTR_unmixed_rotated.pdf}
		\end{column}
		\hspace{-15mm}
		\begin{column}{0.6\textwidth}
			\begin{itemize}
				\vspace{-3mm}
				\item	At any given time, $t$
				\item	$F$: Molar flow rate
				\item	$C$: Concentration
				\item	subscript $j$, $j^\text{th}$ species
				\item	subscript $0$, inlet stream
				\item	subscript $r$, reactor
			\end{itemize}
		\end{column}
	\end{columns}
	$C_{jr}$: Concentration of species j in the reactor
\end{frame}
\end{comment}

\begin{frame}\frametitle{Continuous-Stirred Tank Reactor}
	\begin{itemize}
		\item	CSTR's: are assumed to be well mixed
		\item	System properties constant throughout reactor.
		\item	This implies the product concentrations, temperatures, and other \textbf{intensive} properties \emph{leaving the tank} are the same as that \emph{within} tank.		
	\end{itemize}
\end{frame}

\begin{frame}\frametitle{Continuous-Stirred Tank Reactor (CSTR)}
	\begin{center}
		\includegraphics[height=0.95\textheight]{\imagedir/reactors/examples/flickr-6240399099_97069db83c_o.jpg}
	\end{center}
	\vspace{-60pt}
	\see{Flickr \href{http://www.flickr.com/photos/acwa/6240399099}{\#6240399099}}
\end{frame}

\begin{frame}\frametitle{Continuous-Stirred Tank Reactor}
	\begin{itemize}
		
		\item	Where are CSTRs used? When we want entire system to be operated:
		\begin{itemize}
			\item	at the same concentration
			\item	at the same temperature
		\end{itemize}
		\item	where we require good agitation to contact the reactants
		\item	e.g. emulsion polymerization
		\item	a catalyst is suspended in a liquid product
		\item	leaching gold from crush ore (rock); crushed particles $\sim 50 \mu m$ with gold particles exposed
		\item	Leaching: $4\text{Au} + 8\text{NaCN} + \text{O}_2 + 2 \text{H}_2\text{O} \longrightarrow  4 \text{Na[Au(CN)}_2\text{]} + 4 \text{NaOH}$
	\end{itemize}
\end{frame}

% 14 January 2013
\begin{frame}\frametitle{Continuous-Stirred Tank Reactor (CSTR)}
	\begin{columns}[c]
		\column{0.55\textwidth}
			\includegraphics[width=\textwidth]{\imagedir/reactors/CSTRs/CSTR-notation.png}
		\column{0.60\textwidth}
			\begin{itemize}
				\vspace{-3mm}
				\item	At any given time, $t$
				\item	$F$: molar flow rate
				\item	$C$: concentration
				\item	$v$: volumetric flow
				\item	subscript $j$, $j^\text{th}$ species
				\item	subscript $0$, inlet stream
				%\item	subscript $r$, reactor
			\end{itemize}
	\end{columns}
	\vspace{12pt}
	$C_{j}$: Concentration of species $j$ in the reactor of liquid volume $V$
	\vspace{6pt}
	\begin{itemize}
		\item	$V$ is \textbf{not} the total physical reactor volume
	\end{itemize}
\end{frame}

\begin{frame}\frametitle{Continuous-Stirred Tank Reactor (CSTR)}
	\begin{itemize}
		\item	General mole balance equation: $$F_{j0} - F_j + \int_V r_j (V) dV = \frac{dN_j}{dt}$$
		\vspace{12pt} 
		\item	$\dfrac{dN_j}{dt} = 0$, (often analyzed at steady state, {\bf but not necessarily)}. 
		\vspace{12pt}
		\item	No spatial variation with reactors $$\Rightarrow \int_V r_j(V)dV = r_jV\Rightarrow \boxed{V = \frac{F_{j0} - F_j}{-r_j}}$$ 
	\end{itemize}
\end{frame}

\begin{frame}\frametitle{Continuous-Stirred Tank Reactor (CSTR)}
	We can relate to concentration via
	\vspace{12pt}
	\begin{columns}[t]
		\column{0.50\textwidth}
			\textbf{Awkward Fogler notation}
			\begin{align*}
				F_j &= C_jv\\
				\frac{\text{moles}}{\text{time}} &= \frac{\text{moles}}{\text{vol}}\cdot\frac{\text{volume}}{\text{time}}
			\end{align*}
			$$V = \frac{v_0C_{A0} - vC_A}{-r_A} \eqno{\text{(1-9)}}$$
		\column{0.50\textwidth}
			\textbf{Improved (?)\\ notation}
			\begin{align*}
				F_j &= C_jq\\
				\frac{\text{moles}}{\text{time}} &= \frac{\text{moles}}{\text{vol}}\cdot\frac{\text{volume}}{\text{time}}
			\end{align*}
			$$V = \frac{q_0C_{A0} - qC_A}{-r_A} $$
	\end{columns}
	%Assignment 1 question: Rewrite the mole balance without assuming steady state.
\end{frame}


\subsection{Tubular reactors}

\begin{frame}\frametitle{Tubular reactors}
	\begin{center}
		\includegraphics[width=\textwidth]{\imagedir/reactors/PFR/tubular-reactor-Fogler-1.jpg}
	\end{center}
\end{frame}

\begin{frame}\frametitle{Tubular reactors}
	\begin{center}
		\includegraphics[width=\textwidth]{\imagedir/reactors/PFR/tubular-reactor-Fogler-2.jpg}
	\end{center}
\end{frame}

\begin{frame}\frametitle{Tubular reactors}
	\begin{center}
		\includegraphics[width=\textwidth]{\imagedir/reactors/PFR/tubular-reactor-Fogler-3.png}
	\end{center}
\end{frame}

\begin{frame}\frametitle{Tubular reactors (covered on the board in class)}
	\vspace{-0.5em}
	\begin{itemize}
		\item	Assume turbulent flow $\Rightarrow$ modelled as plug (no radial variation in concentration)
		\item	Referred to as plug-flow reactor (PFR).
		\item	Starting point - general mole balance. $$F_{j0} - F_j + \int_Vr_j(V)dV = \frac{dN_j}{dt} = 0$$ Apply to differential volume: $$F_j|_V - F_j|_{V + \Delta V} + r_j\Delta V = 0$$
	\end{itemize}
\end{frame}

\begin{frame}\frametitle{Tubular reactors (covered on the board in class)}
	Rearranging and dividing by $\Delta V$, $$\left[\frac{F_j|_{V+\Delta V} - F_j|_V}{\Delta V}\right] = r_j$$ Taking limit as $\Delta V \rightarrow 0$ $$\boxed{\frac{dF_j}{dV} = r_j}$$ Integral form: $$V_1 = \int_0^{V_1}dV = \int_{F_{j0}}^{F_{j1}}\frac{dF_j}{r_j}$$
\end{frame}

\begin{frame}\frametitle{Tubular reactors (covered on the board in class)}
	\textbf{Remark 1}: Analysis does not assume constant cross-sectional area.

	$\Rightarrow$ applicable also to other geometries

	End up with same design equation.

	\vspace{24pt}

	\textbf{Remark 2}: Furthermore, for liquids (why not for gases?) at steady state the volumetric flow rate is equal throughout the reactor, i.e., $v_o=v$
\end{frame}

\begin{frame}\frametitle{PFR example}
	$A\longrightarrow B$ in a tubular reactor. Feed enters at constant volumetric rate $q = 10\,\text{L.min}^{-1}$. The reaction follows first-order kinetics with rate const $k = 0.23\,\text{min}^{-1}$. 
	\vspace{12pt}
	\begin{enumerate}
		\item	Determine the volume required to reduce the exiting concentration to $10\%$ of the entering value, i.e. $C_A = 0.1 C_{A0}$?
		\item	As above, but for 99\% conversion, i.e. $C_A = 0.01 C_{A0}$?
		\item	What happens to the conversion if we halve the volumetric flow rate $q_\text{new} \leftarrow q_\text{previous}$. Intuitively, what do we expect?
	\end{enumerate}
\end{frame}

\begin{frame}\frametitle{PFR example}
	\begin{align*}
		\frac{dF_A}{dV} &= r_A = -kC_A
	\end{align*}
	Express design equation in terms of concentration: 
	$$\frac{dF_A}{dV} = \frac{d(C_Aq)}{dV} = q\frac{dC_A}{dA}$$ 
	$$\therefore q\frac{dC_A}{dV} = r_A = -kC_A$$
\end{frame}

\begin{frame}\frametitle{PFR example}
	\begin{align*}
		&\Rightarrow -\frac{q}{k}\int_{C_{A0}}^{C_A}\frac{dC_A}{C_A} = \int_0^VdV\\
		&\Rightarrow -\frac{q}{k}\ln\left(\frac{C_A}{C_{A0}} \right) = V\\
		&\Rightarrow V = -\frac{10}{0.23}\ln\left(\frac{0.1C_{A0}}{C_{A0}} \right) = -\frac{10}{0.23}(-2.3) = 100\,\text{L}
	\end{align*}
	Note: Plug in the numbers at the very last step
	\begin{enumerate}
		\setcounter{enumi}{1}
		\item	For 99\% conversion: 200L
		\item	For half input flow: 99\% conversion {\color{myGreen}{\small (increased!)}}
	\end{enumerate}
	\vspace{4pt}
	\hrule
	\vspace{4pt}
	\scriptsize See F2006 example 1.1 or F2011 example 1.2 for similar problem
\end{frame}

% \begin{frame}\frametitle{Problem 1.4}
% 	Consider a PFR (total volume $V_{PFR}$) with an inlet volumetric flow rate $v_0$ and concentration of A at $C_{A0}$, and an additional volumetric flow rate per Volume (along the length of the reactor) of $\tilde{v}_0$ at a concentration of A at $C_{A0}$. An elementary reaction with reaction rate $r_A=-kC_A$ takes place in the reactor. Determine the outlet concentration of A assuming constant density.
% \end{frame}
% 
% \begin{frame}\frametitle{Solution 1.4}
% 	A mass balance on the differential volume yields: $$v\rho|_V - v\rho|_{V + \Delta V} + \rho\tilde{v}_0\Delta V = 0$$ i.e., $\frac{dv}{dV}=\tilde{v}_0$\\
% 	A mole balance on the differential volume yields: $$vC_A|_V - vC_A|_{V + \Delta V} + C_{A0}\tilde{v}_0\Delta V -k C_A\Delta V= 0$$ i.e., $\frac{d(vC_A)}{dV}=C_{A0}\tilde{v}_0-kC_A$
% \end{frame}
% 
% \begin{frame}\frametitle{Solution 1.4}
% 	Very important to specify initial conditions: $v(0)=v_0$ and $C_A(0)=C_A(0)$ gives $$v(V)=v_0+\tilde{v}_0V$$ $$\frac{vdC_A}{dV}+{C_A\tilde{v}_0}=C_{A0}\tilde{v}_0-kC_A$$ i.e., $(v_0+\tilde{v}_0V)\frac{dC_A}{dV}=(C_{A0}-C_A)\tilde{v}_0-kC_A$
% 
% 	Assignment 1 question-Integrate to get $C_A(V)$
% \end{frame}

\subsection{Packed-Bed Reactor}
\begin{frame}\frametitle{Packed-bed reactor (PBR)}
	Analogous development, except
	\begin{itemize}
		\item	we use $W$ (mass of catalyst) instead of $V$ as independent variable, and
		\item	$r_j' = \left[\frac{\text{moles }j}{\text{(time)(mass catalyst)}}\right]$ instead of

		$r_j =\left[\frac{\text{moles }j}{\text{(time)(volume)}}\right]$
	\end{itemize}
\end{frame}

\begin{frame}\frametitle{Comparison}
	\begin{columns}[t]
		\column{0.50\textwidth}
			\textbf{Plug flow reactor}
			\begin{itemize}
				\item	PFR
				\item	Mole balance at steady state: $$F_{j0} - F_j + \int_Vr_j'dV = 0$$ 
				\item	Differentiate:

				$$\boxed{\frac{dF_j}{dV} = r_j}$$
			\end{itemize}
		\column{0.50\textwidth}
			\textbf{Packed bed reactor}
			\begin{itemize}
				\item	PBR
				\item	Mole balance at steady state: $$F_{j0} - F_j + \int_Wr_j'dW = 0$$
				\item	Differentiate:

				$$\boxed{\frac{dF_j}{dW} = r_j'}$$
			\end{itemize}
	\end{columns}
	
	
\end{frame}

\subsection{Summary}
\begin{frame}\frametitle{Summary}
	General mole balance equation: $$ F_{j0} - F_j + \int_Vr_j(V)dV = \frac{dN_j}{dt} $$
\end{frame}

\begin{frame}\frametitle{Summary}
	\begin{itemize}
		\item[a)]{Batch reactor}

		Differential form $\displaystyle\frac{dN_j}{dt} = r_jV$
		\vspace{24pt}
		\item[b)]{CSTR} $$V = \frac{F_{j0} - F_j}{-r_j}$$
	\end{itemize}
\end{frame}

\begin{frame}\frametitle{Summary}
	\begin{itemize}
		\item[c)]{PFR}

		Differential form: $\displaystyle\frac{dF_j}{dV} = r_j$

		Integral form $V = \displaystyle\int_{F_{j0}}^{F_{jf}}\frac{dF_j}{r_j}$
	\end{itemize}
	
	Given initial and final concentrations, we expressed equations in terms of $C_A$ and solved.
\end{frame}

% \section{Industrial reactors}
% \begin{frame}\frametitle{1.5 Industrial Reactors}
% 	Read Fogler - examinable!
% \end{frame}
