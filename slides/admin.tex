\begin{frame}\frametitle{Plan for today's class}
	\begin{enumerate}
		\item	Background 
		\item	Administrative issues 
		\item	Course contents 
	\end{enumerate}
\end{frame}

\begin{frame}\frametitle{Background}
	{\color{myGreen}{About myself}}
	\begin{itemize}
		\item	Undergraduate degree from University of Cape Town, 1999
		\item	Masters degree from McMaster, 2002 (not a ``doctor'', please)
		\item	Worked with a number of companies since then on data analysis and consulting projects
		\item	Worked at GSK on a 1-year contract until June 2012
		\item	Now working full-time at McMaster since July 2012
		\item	Drop-in hours: Wednesday AM, Thursday PM, Friday PM
		\item	Office is in BSB, room B105
		\item	Arrange a meeting: \url{kevin.dunn@mcmaster.ca}
		\item	Cell: (905) 921 5803 and not \sout{extension 27337}
	\end{itemize}	
\end{frame}

\begin{frame}\frametitle{Acknowledgments}
	Dr. Prashant Mhaskar
	\begin{itemize}
		\item	Most of the class notes (slides) we will use are his
		\item	Today's slides are mine; but we will use his in the next classes
	\end{itemize}
\end{frame}

\begin{frame}\frametitle{Administrative issues}
	\begin{itemize}
		\item	TA introductions 
		\item	Announcement 
		\item	Video and audio 
		\item	Website 
		\item	References 
		\item	Software 
		\item	Expectations 
		\item	Grading 
	\end{itemize}
\end{frame}

\begin{frame}\frametitle{Teaching assistants}
	\vspace{12pt}
	{\color{myGreen}{Vida Meidanshahi}}
	\begin{itemize}
		\item	\url{meidanv@mcmaster.ca}
		\item	JHE, room 141/A
		\item	extension 27342
		\item	Currently doing her Ph.D with Tom Adams
	\end{itemize}
	
	\vspace{24pt}
	
	{\color{myGreen}{Dominik Seepersad}}
	\begin{itemize}
		\item	\url{seeperd@mcmaster.ca}
		\item	JHE, room 370
		\item	extension 22008
		\item	Currently doing his M.A.Sc with Tom Adams
	\end{itemize}
	\vspace{24pt}
	Office hours: by email appointment
\end{frame}

\begin{frame}\frametitle{Video and audio}
	\begin{itemize}
		\item	Available for all my courses
		\item	Purpose: for your review, and to prepare for assignments, tutorials and exams 
		\item	Might be useful if you miss a class
		\item	As long as feasible, I will try to video record all classes
		\begin{itemize}
			\item	Try to record just myself, the board and the projector 
			\item	Can't guarantee the quality will be very good (background noise, etc) 
			\item	Video should be available within 24 to 48 hours after the class 
		\end{itemize}
		\item	Audio recordings will be made available as well 
	\end{itemize}
\end{frame}

\begin{frame}\frametitle{Course website}
	\begin{exampleblock}{}
		\centering 
		\href{http://learnche.mcmaster.ca/3K4}{http://learnche.mcmaster.ca/3K4}
	\end{exampleblock}
	\begin{itemize}
		\item	Slides will be added to the site before class
		\item	Please print slides and bring to class
		\item	Assignments and tutorials will be posted there
		\item	Solutions to selected/most problems will be available
	\end{itemize}
	\vspace{12pt}
	\textbf{Website is the main reference for all things course-related} 
	\begin{itemize}
		\item	expected to check it about 3 times per week {\tiny (top left)}
		\item	or follow on Twitter to get updates: \href{https://twitter.com/3k4reactors}{@3k4reactors}
	\end{itemize}
\end{frame}

\begin{frame}\frametitle{Reference text book}
	
	\begin{columns}[t]
		\column{0.80\textwidth}
			\emph{Required}: Fogler, ``\textbf{Essentials} of Chemical Reaction Engineering'', (1st edition)
			
			\vspace{3pt}
			\texttt{F2011}
		\column{0.10\textwidth}
			\vspace{-1cm}
			\begin{center}
				\includegraphics[width=1.5\textwidth]{\imagedir/reactors/textbook-covers/Fogler-Essentials-CRE.jpg}
			\end{center}
	\end{columns}
	\vspace{0pt}
	\begin{center}
		\emph{\textbf{or}}
	\end{center}
	\vspace{12pt}
	\begin{columns}[t]
		\column{0.80\textwidth}
			\emph{Required}: Fogler, ``\textbf{Elements} of Chemical Reaction Engineering'', (4th edition)
			
			\vspace{3pt}
			\texttt{F2006}
		\column{0.10\textwidth}
			\vspace{-1cm}
			\begin{center}
				\includegraphics[width=1.5\textwidth]{\imagedir/reactors/textbook-covers/Fogler-Elements-CRE.jpg}
			\end{center}
	\end{columns}
	
	\begin{itemize}
		\item	Some other suggested books are on the course website: 
		\begin{itemize}
			\item	Self-directed learning
		\end{itemize}
	\end{itemize}
\end{frame}

\begin{frame}\frametitle{Course feedback via Learning website}
	\begin{itemize}
		\item	I might not have explained something clearly; 
		\item	You didn't get a chance to ask a question, \emph{etc}
	\end{itemize}
	\href{http://learnche.mcmaster.ca/feedback-questions}{http://learnche.mcmaster.ca/feedback-questions}
	\vspace{12pt}
	\hrule
	\begin{center}
		\includegraphics[width=0.65\textwidth]{\imagedir/teaching/anonymous-feedback.png}
	\end{center}
	\hrule
\end{frame}

\begin{frame}\frametitle{Course software}
	\begin{itemize}
		\item	A computer can be used for later assignments and for the course project 
		\item	I support the use of any language, Python and MATLAB in particular
		\item	Symbolic processing: R and {\small\href{http://integrals.wolfram.com/}{http://integrals.wolfram.com/}}
		\item	Numerical integration: POLYMATH
	\end{itemize}
	\vspace{12pt}
	Let me know if you find other useful tools
\end{frame}

\begin{frame}\frametitle{Expectations outside class}
	\begin{itemize}
		\item	You can expect TAs and I to answer emails promptly 
		\item	If you have questions: 
		\begin{enumerate}
			\item	Please email the TAs with CC to me \hfill {\tiny{\color{myOrange}{$\longleftarrow$ hopefully this solves your problem}}}
			\item	Please send from your McMaster address
			\item	Set up in-person meeting with TAs or myself 
			\item	My office hours: Wednesday AM, Thursday PM and Friday PM
		\end{enumerate}
	\end{itemize}
\end{frame}

\begin{frame}\frametitle{Why study reactor design?}
	\begin{itemize}
		\item	It is an unique course to Chemical Engineering
			\begin{itemize}
				\item	Reactor Design $\qquad\,\,$[$\sim 25$\% of capital costs]
				\item	Separation Processes [$\sim 75$\% of capital costs]
			\end{itemize}
			\vspace{12pt}
			while other engineers and scientists also study:
			\begin{itemize}
				\item	numerical methods
				\item	simulation and modelling
				\item	thermodynamics
				\item	fluid flow and heat transfer
				\item	statistics
				\item	process control
				\item	bioprocessing
				\item	polymers
				\item	problem solving
			\end{itemize}
	\end{itemize}
	But this is not a good justification.
\end{frame}

\begin{frame}\frametitle{Why study reactor design?}
	\begin{exampleblock}{}
		\begin{center}
			{\color{myGreen}{Chemical Engineering is about ``processing'' material}}
		\end{center}
	\end{exampleblock}
	and at the heart of any processing system is usually a reaction.
	
	\vspace{12pt}
	\begin{itemize}
		\item	Reactors are some of the least impressive looking units
		\item	However, the entire plant economics and profitability are dependent on the reactor
		\item	Get the reactor design wrong and the entire plant can be a failure
		\item	Plenty of time spent on its design and optimization (years ... not months are often taken)
	\end{itemize}	
\end{frame}

\begin{frame}\frametitle{Why study reactor design?}
	The outcome of a ``reactor design'': multi-compartment autoclave
	\begin{center}
		\includegraphics[width=\textwidth]{\imagedir/reactors/examples/reactor-design-GMD-4-DETACLAD-Explosion-Clad.png}
		
		4.7m inside diameter, 25m long
	\end{center}
	\see{Hydrometallurgy of Nickel and Cobalt 2009, The Metallurgical Society of the Canadian Institute of Mining, Metallurgy and Petroleum, Symposium}
	% Published in Hydrometallurgy of Nickel and Cobalt 2009, Proceedings of the International Symposium 39th Annual Hydrometallurgy Meeting 
	% The Metallurgical Society of the Canadian Institute of Mining, Metallurgy and Petroleum, Pages 181-193, 
	% DETACLAD Explosion Clad for Autoclaves and Vessels, John G.Banker
\end{frame}

\begin{frame}\frametitle{Why study reactor design?}
	But most usefully, and the most likely case you will find: 
	\begin{exampleblock}{}
		You are working at an existing plant, with the reactor running for many years and producing product profitably.
	\end{exampleblock}
	\vspace{12pt}
	\begin{enumerate}
		\item	A recession hits (e.g. 2007/2008/2009), and you need to scale back. Less demand for your product.
		\begin{itemize}
			\item	What reduced flow rate do you operate at?
			\item	Will the reaction still go the required conversion?
		\end{itemize}
		\vspace{12pt}
		\item	There is increased demand for your product, right away. No time to design, install and commission a second reactor.
		\begin{itemize}
			\item	What increased flow rate do we operate at?
			\item	What will happen to side-reactions and impurities?
			\item	Can we still get the same conversion in the fixed-size reactor?
			\item	Maybe change the catalyst, or operate at a higher temperature?
		\end{itemize}
	\end{enumerate}
\end{frame}

\begin{frame}\frametitle{Why study reactor design?}
	\begin{enumerate}
		\setcounter{enumi}{2}
		\item	Competitors from another country/company/using a different technology are making the same product, at a cheaper price.
		\begin{itemize}
			\item	Can you operate at lower temperatures to save money?
			\item	What can you adjust to increase conversion
			\item	How can you further reduce impurities in the outlet, creating a more valuable product for your customer?
		\end{itemize}
		\begin{center}
			\includegraphics[width=.75\textwidth]{\imagedir/examples/competitor-product/competitor-product.png}
		\end{center}
	\end{enumerate}
\end{frame}

\begin{frame}\frametitle{What is ``Reactor Design''?}
	\begin{exampleblock}{}
		How to produce a {\color{purple}{specified product}}, at a {\color{orange}{given rate}}, from {\color{myGreen}{known reactants}}.
	\end{exampleblock}
	\vspace{12pt}
	\begin{columns}[t]
		\column{0.50\textwidth}
			\textbf{Decisions we have to make}
			\begin{itemize}
				\item	What type of reactor do we use? Batch, plug-flow, tank, packed-bed, others...
				\item	Will we operate isothermally, adiabatically, or a hybrid?
				\item	At what temperature, pressure, compositions and flows do we operate?
				\item	Which phases are present: liquid, solid, gas, or a hybrid?
			\end{itemize}

		\column{0.50\textwidth}
			\textbf{What we need to know}
			\begin{itemize}
				\item	The size of the reactor
				\item	Composition of products leaving
				\item	Temperatures in the reactor
				\item	Pressure in the reactors, and pressure drop across it
			\end{itemize}
	\end{columns}
\end{frame}

\begin{frame}\frametitle{What we won't be studying: ``Design of reactors''}
	\begin{center}
		\includegraphics[width=\textwidth]{\imagedir/reactors/examples/reactor-design-GMD-2.png}
	\end{center}
	\vspace{-12pt}
	\see{\emph{Thanks Dad!\,}}
\end{frame}

\begin{frame}\frametitle{What we won't be studying: ``Impellers and internals''}
	\begin{center}
		\includegraphics[width=\textwidth]{\imagedir/reactors/examples/reactor-design-GMD-3.png}
	\end{center}
	\see{\emph{Thanks Dad!\,}}
\end{frame}

\begin{frame}\frametitle{What we won't be studying: ``Materials of construction''}
	\begin{columns}[t]
		\column{0.50\textwidth}
			\begin{center}
				\includegraphics[width=\textwidth]{\imagedir/reactors/examples/flash-vessel-GMD.jpg}
			\end{center}
		\column{0.50\textwidth}
			\begin{center}
				\includegraphics[width=\textwidth]{\imagedir/reactors/examples/GMD-brick-lined-reactor.pdf}
			\end{center}
	\end{columns}
	\see{\emph{Thanks Dad!\,}}
\end{frame}

\begin{frame}\frametitle{What we won't be studying: ``Internals''}
	\begin{columns}[t]
		\column{0.30\textwidth}
			\begin{center}
				\includegraphics[width=\textwidth]{\imagedir/reactors/examples/autoclave-schematic.png}
			\end{center}
		\column{0.70\textwidth}
			\begin{center}
				\includegraphics[width=\textwidth]{\imagedir/reactors/examples/autoclave-internals-GMD.png}
			\end{center}
	\end{columns}
	\see{\emph{Thanks Dad!\,}}
\end{frame}

\begin{frame}\frametitle{What we won't be studying: ``Externals''}
	\begin{columns}[t]
		\column{0.50\textwidth}
			\begin{center}
				\includegraphics[width=\textwidth]{\imagedir/reactors/examples/autoclave-mixer-and-sensor-GMD.png}
			\end{center}
		\column{0.50\textwidth}
			\begin{center}
				\includegraphics[width=\textwidth]{\imagedir/reactors/examples/reactor-GMD-6.JPG}
			\end{center}
	\end{columns}
	\see{\emph{Thanks Dad!\,}}
\end{frame}

\begin{frame}\frametitle{What we won't be studying}
	\begin{itemize}
		\item	Design of the internals: heating/cooling tubing, flanges, baffles, motors, flanges, clean-in-place systems
		\item	Design of systems with catalysis and mass-transfer limitations
	\end{itemize}
	\vspace{-12pt}
	\begin{columns}[t]
		\column{0.50\textwidth}
			\begin{center}
		\includegraphics[width=\textwidth]{\imagedir/reactors/examples/becomix/inside-becomix.jpg}
	\end{center}
		\column{0.50\textwidth}
			\begin{center}
				\includegraphics[width=0.7\textwidth]{\imagedir/reactors/examples/flickr-6240399099_97069db83c_o.jpg}
				%\includegraphics[width=\textwidth]{\imagedir/reactors/examples/flickr-6276099367_4254173bb3_o-CC-BY-2.0-smaller.jpg}
			\end{center}
			\see{\emph{Above}: \href{http://www.flickr.com/photos/acwa/6240399099/}{flickr:\#6240399099}} 
			
			\see{\emph{Left}: reactor in a clean processing facility} 
	\end{columns}	
\end{frame}

\begin{frame}\frametitle{What we won't be studying}
	\begin{itemize}
		\item	Non-ideal behaviour that is present in most systems 
		\begin{itemize}
			\item	though we will introduce it near the end of the course
			\item	Does this CSTR obey all the assumptions?
		\end{itemize}
	\end{itemize}
	\begin{center}
		\includegraphics[width=\textwidth]{\imagedir/reactors/examples/autoclave-GMD.png}
	\end{center}
\end{frame}

\begin{frame}\frametitle{How this course is structured}
	There are 8 main sections, spread over 12 weeks
	\begin{enumerate}
		\item	Mole balances 
		\item	Conversion and reactor sizing
		\item	Rate laws and stoichiometry
		\item	Isothermal reactor design
		\item	Collection and analysis of rate data
		\item	Multiple reactions
		\item	Steady-state nonisothermal reactor design 
		\item	Distributions of residence times 
	\end{enumerate}
\end{frame}

\begin{frame}\frametitle{Grading}
	What we look for in the grading is demonstration that you/group: 
	\begin{enumerate}
		\item	understand the concept, 
		\item	have the ability to apply the concept to new instances, 
		\item	think creatively about problems, 
		\item	my questions are seldom ``plug-and-chug'',
		\item	numerical accuracy, 
		\item	grammar and spelling.
	\end{enumerate}
\end{frame}

\begin{frame}\frametitle{Grading for assignments}
	\begin{itemize}
		\item	\emph{Appropriate} group work is highly encouraged 
		\begin{itemize}
			\item	Up to 30\% of course grade 
			\item	\emph{Learn with each other} 
			\item	Assignments and project done in groups of 2, or by yourself 
			\item	Hand these in assignments as one submission 
		\end{itemize}
		\item	Late grading 
		\begin{itemize}
			\item	\( -30 \)\% per day 
			\item	2 ``late day'' credits for assignments 
			\item	solutions posted after $\approx 2$ days of due date 
		\end{itemize}
		\item	Assignment grading: 
		\begin{itemize}
			\item	No make-ups for assignments 
			\item	Counts \textbf{15\%} of course grade 
			\item	All assignments will be used for the grade
		\end{itemize}
		\item	Assignment dates: see website
		\item	\href{http://learnche.mcmaster.ca/calendar}{http://learnche.mcmaster.ca/calendar}
	\end{itemize}
\end{frame}

\begin{frame}\frametitle{Group-based assignments}
	\begin{itemize}
		\item	``Appropriate'' group work is highly encouraged (about 30\% of course)
		\item	Learn with each other: groups of 2, no larger
		\vspace{2pt}\hrule\vspace{2pt}
		\item	Optimal group work: \emph{an example of one approach}
			\begin{itemize}
				\item	Sarah and Brad work on an assignment
				\item	Both Sarah and Brad do {\color{myRed}{\textbf{all questions}}} in draft: quick notes at home, on the bus, \emph{etc}, $\pm 4$ days before assignment due
				\pause
				\item	Meet in the library next day and go over each other's notes
				\item	Explain to the other why you disagree
				\item	e.g. Sarah sees a mistaken interpretation in Brad's work
				\begin{itemize}
					\item	She explains why it is a mistake to Brad: Sarah learns
					\item	Brad also learns: he's heard this in class, and from Sarah now
					\item	If neither can resolve it? speak with TAs or Kevin
				\end{itemize}
				\pause
				\item	Write up a joint solution; e.g. Sarah Q1 and 2, Brad does Q3
				\item	Both review it before submitting
			\end{itemize}
		\vspace{2pt}\hrule\vspace{2pt}
		\pause		
		\item	Other approaches are possible: your group decides
		\item	\color{myOrange}{What doesn't work}: Sarah does Q1 and Q2, Brad does Q3; staple and submit; no group review
		\item	\textbf{Do not share files or written work} \emph{between} groups 
	\end{itemize}
\end{frame}

\begin{frame}\frametitle{Grading for exams and project}
	\begin{itemize}
		\item	Written midterm on 13 February: \textbf{20\%} 
		\begin{itemize}
			\item	it is optional 
			\item	there is no make-up 
			\item	if you miss it, your final counts more
			\item	covers all material up to 12 February
		\end{itemize}
	\end{itemize}
	\begin{itemize}
		\item	Written final exam: \textbf{50\%} 
		\begin{itemize}
			\item	Covers all material 
		\end{itemize}
	\end{itemize}
	\begin{itemize}
		\item	Midterm and final exam: 
		\begin{itemize}
			\item	Open notes -- anything on paper is allowed 
			\item	No electronic devices unfortunately 
			\item	Any calculator 
		\end{itemize}
	\end{itemize}
	\begin{itemize}
		\item	Project due on 27 March: \textbf{15\%} 
		\begin{itemize}
			\item	Requires computer software 
			\item	Can collaborate, \textbf{but only within your group}: not between groups 
		\end{itemize}
	\end{itemize}
\end{frame}

\begin{frame}\frametitle{Electronic submissions}
	You may submit assignments electronically
	\begin{itemize}
		\item	via Google Docs only
		\item	Share your document with special gmail addresses:
		\begin{itemize}
			\item	\url{kgdunn@gmail.com}
			\item	\url{mac.che.3k4@gmail.com}
		\end{itemize}
		\item	More details on the course website
		\item	TAs will grade it electronically
		\item	Saves paper, allows group collaboration
	\end{itemize}
\end{frame}

\begin{frame}\frametitle{Important dates}
	\begin{itemize}
		\item	13 February mid-term  %T28 room 1, 19:00 to 21:00 
		\item	27 March: project due 
		\item	08 April: last (review) class
		\item	April: final-exam 
	\end{itemize}
	\begin{itemize}
		\item	Something due every 2nd week: see calendar
	\end{itemize}
\end{frame}

